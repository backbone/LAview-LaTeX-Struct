%% LyX 2.0.3 created this file.  For more info, see http://www.lyx.org/.
%% Do not edit unless you really know what you are doing.
\documentclass[english,russian]{article}
\usepackage[T1]{fontenc}
\usepackage[utf8x]{inputenc}
\usepackage{geometry}
\geometry{verbose}
\usepackage{array}
\usepackage{longtable}
\usepackage{textcomp}
\usepackage{amstext}
\usepackage{graphicx}

\makeatletter

%%%%%%%%%%%%%%%%%%%%%%%%%%%%%% LyX specific LaTeX commands.
\DeclareRobustCommand{\cyrtext}{%
  \fontencoding{T2A}\selectfont\def\encodingdefault{T2A}}
\DeclareRobustCommand{\textcyr}[1]{\leavevmode{\cyrtext #1}}
\AtBeginDocument{\DeclareFontEncoding{T2A}{}{}}

\newcommand{\lyxmathsym}[1]{\ifmmode\begingroup\def\b@ld{bold}
  \text{\ifx\math@version\b@ld\bfseries\fi#1}\endgroup\else#1\fi}

%% Because html converters don't know tabularnewline
\providecommand{\tabularnewline}{\\}

%%%%%%%%%%%%%%%%%%%%%%%%%%%%%% User specified LaTeX commands.
\usepackage{multirow}



\usepackage{babel}






\usepackage{babel}






\usepackage{babel}





\usepackage{babel}


\makeatother

\usepackage{babel}
\begin{document}
\begin{flushright}
ф. 07-726А 
\par\end{flushright}

\begin{center}
\textbf{Формулярные данные по 06В.00.0100ВТУ} 
\par\end{center}

\begin{center}
двигателя ТВ7-117В №\rule[-1pt]{3.5cm}{0.4pt} Температура наружного
воздуха $t_{\text{н}}$, \rule[-1pt]{1cm}{0.4pt}, $\,^{\circ}\mbox{C}$ 
\par\end{center}

от\_\_\_\_\_\_\_\_\_\_ <<\rule[-1pt]{1.1cm}{0.4pt}>> \rule[-1pt]{2.5cm}{0.4pt}
20~~~~г.~~~~~~Давление наружного воздуха, $P_{\text{Н}}$\rule[-1pt]{1cm}{0.4pt}
мм. рт. ст.

Drossel55.Table1

\begin{longtable}{|>{\raggedright}m{0.11\paperwidth}|>{\centering}m{0.04\paperwidth}|>{\centering}m{0.1\paperwidth}|>{\centering}m{0.09\paperwidth}|>{\centering}m{0.09\paperwidth}|>{\centering}m{0.1\paperwidth}|>{\centering}m{0.09\paperwidth}|}
\hline 
 \newpage  
\multirow{2}{0.11\paperwidth}{\centering{}\linebreak{}
 Режим работы двигателя}  &  & \multirow{2}{0.1\paperwidth}{\textbf{\centering{}}\linebreak{}
 Мощность на выводном валу \textbf{N л.с.} }  & \multicolumn{2}{c|}{Частота вращения ротора} & \multirow{2}{0.1\paperwidth}{\centering{}Температура газа
перед свободнойтурбиной не более \textbf{t$_{4}$},$\,^{\circ}\mbox{C}$}  & \multirow{2}{0.09\paperwidth}{\centering{}Удельный расход
топлива не более \textbf{C$_{R\text{ пр}}$}г/л.с.ч.}\tabularnewline
\hline 
 &  &  & турбо-компрессора не более \textbf{n$_{\text{тк пр}}$}, \%  & свободной турбины \textbf{n$_{\lyxmathsym{ст}}$}, \%  &  & \tabularnewline
\hline 
\multirow{2}{0.11\paperwidth}{{*}2,5-минутной мощности}  & ТУ  &  & 101,0  & 98$\pm$0,5  & 805  & -\tabularnewline
\hline 
 & Факт  &  &  &  &  & \tabularnewline
\hline 
\multirow{2}{0.11\paperwidth}{30-минутной мощности}  & ТУ  & 3000  & 98,5  & 98$\pm$0,5  & 795  & -\tabularnewline
\hline 
 & Факт  &  &  &  &  & \tabularnewline
\hline 
\multirow{2}{0.11\paperwidth}{ВЗЛ}  & ТУ  & 2800  & 97,5  & 98$\pm$0,5  & 775  & 205\tabularnewline
\hline 
\newpage & Факт  &  &  &  &  & \tabularnewline
\hline 
\multirow{2}{0.11\paperwidth}{МП}  & ТУ  & 2100  & 94,0  & 98$\pm$0,5  & 705  & -\tabularnewline
\hline 
 & Факт  &  &  &  &  & \tabularnewline
\hline 
\multirow{2}{0.11\paperwidth}{МКр}  & ТУ  & 1900  & 93,0  & 98$\pm$0,5  & 680  & 221\tabularnewline
\hline \newpage
 & Факт  &  &  &  &  & \tabularnewline
\hline 
\multirow{2}{0.11\paperwidth}{{*} ЗМГ}  & ТУ  & -  & -  & 80$\pm$0,5  & -  & -\tabularnewline
\hline 
 & Факт  &  &  &  &  & \tabularnewline
\hline 
\end{longtable}

\begin{flushleft}
{*} - измеренные параметры\hspace*{1in}\textbf{\textsc{Настроечные
параметры}} 
\par\end{flushleft}

\begin{flushleft}
Drossel55.Table2\foreignlanguage{english}{}%
\begin{longtable}{|>{\centering}m{0.15\paperwidth}|>{\centering}m{0.15\paperwidth}|>{\centering}m{0.1\paperwidth}|>{\centering}m{0.1\paperwidth}|>{\raggedright}m{0.19\paperwidth}|}
\hline 
\multicolumn{2}{|>{\centering}m{0.2\paperwidth}|}{\centering{}\textbf{Параметр} } & \multirow{2}{0.1\paperwidth}{\textbf{\centering{}Обозначение}}  & \multirow{2}{0.1\paperwidth}{\textbf{\centering{}Режим}}  & \multirow{2}{0.1\paperwidth}{\textbf{\centering{}Значение,
полученное при испытании}} \tabularnewline
\hline 
\multicolumn{2}{|>{\centering}p{0.2\paperwidth}|}{ \textbf{Наименование и единица измерения} } &  &  & \tabularnewline
\endhead
\hline\newpage
\multicolumn{2}{|c|}{ 1 } & 2  & 3  & 4 \tabularnewline
\hline 
\multicolumn{2}{|>{\centering}p{0.2\paperwidth}|}{ %
\begin{minipage}[t]{0.45\columnwidth}%
Настроечная величина частоты вращения

ротора турбокомпрессора, \%%
\end{minipage}} & \centering{}$n_{\text{тк\_мг}}0$  & МГ  & \tabularnewline
\hline 
\multicolumn{2}{|>{\centering}p{0.08\paperwidth}|}{ %
\begin{minipage}[t]{0.45\columnwidth}%
Настроечная величина частоты вращения

ротора турбокомпрессора, \%%
\end{minipage}} & \centering{}$n_{\text{тк\_взл}}0$  & <<Взлёт>>  & \tabularnewline
\hline 
\multicolumn{2}{|c|}{ %
\begin{minipage}[c]{0.45\columnwidth}%
Настроечная величина минимальной частоты вращения ротора турбокомпрессора,
\%%
\end{minipage}} & \centering{}$n_{\text{тк\_мин}}0$  & МГ  & \tabularnewline
\hline\newpage 
\multicolumn{2}{|c|}{ %
\begin{minipage}[t]{0.45\columnwidth}%
Настроечная величина перестройки частоты вращения ротора турбокомпрессора
при <<Разрешении РПМ>>,\%%
\end{minipage}} & \centering{}$\Delta n_{\text{тк\_взл}}0$  & <<Взлёт>>  & \tabularnewline
\hline 
\newpage\multicolumn{2}{|>{\centering}p{0.08\paperwidth}|}{ %
\begin{minipage}[t]{0.45\columnwidth}%
Настроечная величина перестройки частоты вращения ротора турбокомпрессора
по команде <<Тренировочный режим>>, \%%
\end{minipage}} & \centering{}$\Delta n_{\text{тк\_тр}}0$  & <<Взлёт>>  & \tabularnewline
\hline 
\multicolumn{2}{|c|}{ %
\begin{minipage}[t]{0.45\columnwidth}%
Расходный коэффициент 1-го контура форсунок%
\end{minipage}} & \centering{}$\Psi_{(\mu F)}$  & -  & \tabularnewline
\hline 
\multicolumn{2}{|>{\centering}p{0.1\paperwidth}|}{ %
\begin{minipage}[t]{0.45\columnwidth}%
Программный расход топлива в РС, кг/ч%
\end{minipage}} & \centering{}$G_{\text{т прог РС МГ}}$  & -  & \tabularnewline
\hline 
\multirow{2}{0.15\paperwidth}{Параметры измерителя крутящего
момента\foreignlanguage{english}{  } } & Начальное смещение, град  & \centering{}$\phi_{\text{икм}}0$  & \multirow{2}{0.1\paperwidth}{\centering{}-}  & \tabularnewline
\hline 
 & Коэффициент характеристики  & \centering{}$K$  &  & \tabularnewline
\hline 
\end{longtable}
\par\end{flushleft}

Расход масла, л/ч \rule[-1pt]{2cm}{0.4pt} (не более 0,2)

\begin{flushleft}
\textbf{Зависимость между положением дозирующей иглы и расходом топлива
$G_{\text{т}}=f(\alpha_{\text{дк}})$} 
\par\end{flushleft}

\begin{flushleft}
Drossel55.Table3 
\par\end{flushleft}

\begin{flushleft}
\begin{longtable}{|>{\centering}p{0.15\paperwidth}|>{\centering}p{0.1\paperwidth}|>{\centering}p{0.03\paperwidth}|>{\centering}p{0.03\paperwidth}|>{\centering}p{0.03\paperwidth}|>{\centering}p{0.03\paperwidth}|>{\centering}p{0.03\paperwidth}|>{\centering}p{0.03\paperwidth}|>{\centering}p{0.03\paperwidth}|>{\centering}p{0.03\paperwidth}|>{\centering}p{0.03\paperwidth}|>{\centering}p{0.03\paperwidth}|}
\toprule 
\multicolumn{2}{|c|}{Параметр} & \multicolumn{10}{c|}{Значение, полученное при испытании}\tabularnewline
\midrule 
Наименование и единица измерения  & Обозначение  & 1  & 2  & 3  & 4  & 5  & 6  & 7  & 8  & 9  & 10\tabularnewline
\midrule 
   \newpage  Положение дозирующей иглы, град  & $(\alpha_{\text{дк}})$  &  &  &  &  &  &  &  &  &  & \tabularnewline
\midrule 
Расход топлива, кг/ч  & $G_{\text{Т}}$  &  &  &  &  &  &  &  &  &  & \tabularnewline
\midrule 
\newpage Исполнитель  & Инженер по испытаниям  & \multicolumn{3}{c|}{%
\begin{minipage}[t]{0.1\columnwidth}%
Начальник БТК ИК%
\end{minipage}} & \multicolumn{4}{>{\centering}p{0.15\paperwidth}|}{%
\begin{minipage}[t]{0.15\columnwidth}%
Начальник участка ИД%
\end{minipage}} & \multicolumn{3}{c|}{%
\begin{minipage}[t]{0.15\columnwidth}%
Представитель заказчика%
\end{minipage}}\tabularnewline
\midrule 
 &  & \multicolumn{3}{c|}{} & \multicolumn{4}{c|}{} & \multicolumn{3}{c|}{}\tabularnewline
\bottomrule 
\end{longtable}
\par\end{flushleft}

\begin{tabular}{|>{\raggedright}m{0.11\paperwidth}|>{\centering}m{0.04\paperwidth}|>{\centering}m{0.1\paperwidth}|>{\centering}m{0.09\paperwidth}|>{\centering}m{0.09\paperwidth}|>{\centering}m{0.1\paperwidth}|>{\centering}m{0.09\paperwidth}|}
\hline 
 \newpage  
\multirow{2}{0.11\paperwidth}{\centering{}\linebreak{}
 Режим работы двигателя}  &  & \multirow{2}{0.1\paperwidth}{\textbf{\centering{}}\linebreak{}
 Мощность на выводном валу \textbf{N л.с.} }  & \multicolumn{2}{c|}{Частота вращения ротора} & \multirow{2}{0.1\paperwidth}{\centering{}Температура газа
перед свободнойтурбиной не более \textbf{t$_{4}$},$\,^{\circ}\mbox{C}$}  & \multirow{2}{0.09\paperwidth}{\centering{}Удельный расход
топлива не более \textbf{C$_{R\text{ пр}}$}г/л.с.ч.}\tabularnewline
\hline 
 &  &  & турбо-компрессора не более \textbf{n$_{\text{тк пр}}$}, \%  & свободной турбины \textbf{n$_{\lyxmathsym{ст}}$}, \%  &  & \tabularnewline
\hline 
\multirow{2}{0.11\paperwidth}{{*}2,5-минутной мощности}  & ТУ  &  & 101,0  & 98$\pm$0,5  & 805  & -\tabularnewline
\hline 
 & Факт  &  &  &  &  & \tabularnewline
\hline 
\multirow{2}{0.11\paperwidth}{30-минутной мощности}  & ТУ  & 3000  & 98,5  & 98$\pm$0,5  & 795  & -\tabularnewline
\hline 
 & Факт  &  &  &  &  & \tabularnewline
\hline 
\multirow{2}{0.11\paperwidth}{ВЗЛ}  & ТУ  & 2800  & 97,5  & 98$\pm$0,5  & 775  & 205\tabularnewline
\hline 
\newpage & Факт  &  &  &  &  & \tabularnewline
\hline 
\multirow{2}{0.11\paperwidth}{МП}  & ТУ  & 2100  & 94,0  & 98$\pm$0,5  & 705  & -\tabularnewline
\hline 
 & Факт  &  &  &  &  & \tabularnewline
\hline 
\multirow{2}{0.11\paperwidth}{МКр}  & ТУ  & 1900  & 93,0  & 98$\pm$0,5  & 680  & 221\tabularnewline
\hline \newpage
 & Факт  &  &  &  &  & \tabularnewline
\hline 
\multirow{2}{0.11\paperwidth}{{*} ЗМГ}  & ТУ  & -  & -  & 80$\pm$0,5  & -  & -\tabularnewline
\hline 
 & Факт  &  &  &  &  & \tabularnewline
\hline 
\end{tabular}

\begin{flushleft}
{*} - измеренные параметры\hspace*{1in}\textbf{\textsc{Настроечные
параметры}} 
\par\end{flushleft}

\begin{flushleft}
Drossel55.Table2\foreignlanguage{english}{}%
\begin{tabular}{|>{\centering}m{0.15\paperwidth}|>{\centering}m{0.15\paperwidth}|>{\centering}m{0.1\paperwidth}|>{\centering}m{0.1\paperwidth}|>{\raggedright}m{0.19\paperwidth}|}
\hline
\hline\newpage
\multicolumn{2}{|c|}{ 1 } & 2  & 3  & 4 \tabularnewline
\hline 
\multicolumn{2}{|>{\centering}p{0.2\paperwidth}|}{ %
\begin{minipage}[t]{0.45\columnwidth}%
Настроечная величина частоты вращения

ротора турбокомпрессора, \%%
\end{minipage}} & \centering{}$n_{\text{тк\_мг}}0$  & МГ  & \tabularnewline
\hline 
\multicolumn{2}{|>{\centering}p{0.08\paperwidth}|}{ %
\begin{minipage}[t]{0.45\columnwidth}%
Настроечная величина частоты вращения

ротора турбокомпрессора, \%%
\end{minipage}} & \centering{}$n_{\text{тк\_взл}}0$  & <<Взлёт>>  & \tabularnewline
\hline 
\multicolumn{2}{|c|}{ %
\begin{minipage}[c]{0.45\columnwidth}%
Настроечная величина минимальной частоты вращения ротора турбокомпрессора,
\%%
\end{minipage}} & \centering{}$n_{\text{тк\_мин}}0$  & МГ  & \tabularnewline
\hline\newpage 
\multicolumn{2}{|c|}{ %
\begin{minipage}[t]{0.45\columnwidth}%
Настроечная величина перестройки частоты вращения ротора турбокомпрессора
при <<Разрешении РПМ>>,\%%
\end{minipage}} & \centering{}$\Delta n_{\text{тк\_взл}}0$  & <<Взлёт>>  & \tabularnewline
\hline 
\newpage\multicolumn{2}{|>{\centering}p{0.08\paperwidth}|}{ %
\begin{minipage}[t]{0.45\columnwidth}%
Настроечная величина перестройки частоты вращения ротора турбокомпрессора
по команде <<Тренировочный режим>>, \%%
\end{minipage}} & \centering{}$\Delta n_{\text{тк\_тр}}0$  & <<Взлёт>>  & \tabularnewline
\hline 
\multicolumn{2}{|c|}{ %
\begin{minipage}[t]{0.45\columnwidth}%
Расходный коэффициент 1-го контура форсунок%
\end{minipage}} & \centering{}$\Psi_{(\mu F)}$  & -  & \tabularnewline
\hline 
\multicolumn{2}{|>{\centering}p{0.1\paperwidth}|}{ %
\begin{minipage}[t]{0.45\columnwidth}%
Программный расход топлива в РС, кг/ч%
\end{minipage}} & \centering{}$G_{\text{т прог РС МГ}}$  & -  & \tabularnewline
\hline 
\multirow{2}{0.15\paperwidth}{Параметры измерителя крутящего
момента\foreignlanguage{english}{  } } & Начальное смещение, град  & \centering{}$\phi_{\text{икм}}0$  & \multirow{2}{0.1\paperwidth}{\centering{}-}  & \tabularnewline
\hline 
 & Коэффициент характеристики  & \centering{}$K$  &  & \tabularnewline
\hline 
\end{tabular}
\par\end{flushleft}

Расход масла, л/ч \rule[-1pt]{2cm}{0.4pt} (не более 0,2)

\begin{flushleft}
\textbf{Зависимость между положением дозирующей иглы и расходом топлива
$G_{\text{т}}=f(\alpha_{\text{дк}})$} 
\par\end{flushleft}

\begin{flushleft}
Drossel55.Table3 
\par\end{flushleft}

\begin{flushleft}
\begin{tabular}{|>{\centering}p{0.15\paperwidth}|>{\centering}p{0.1\paperwidth}|>{\centering}p{0.03\paperwidth}|>{\centering}p{0.03\paperwidth}|>{\centering}p{0.03\paperwidth}|>{\centering}p{0.03\paperwidth}|>{\centering}p{0.03\paperwidth}|>{\centering}p{0.03\paperwidth}|>{\centering}p{0.03\paperwidth}|>{\centering}p{0.03\paperwidth}|>{\centering}p{0.03\paperwidth}|>{\centering}p{0.03\paperwidth}|}
\toprule 
\multicolumn{2}{|c|}{Параметр} & \multicolumn{10}{c|}{Значение, полученное при испытании}\tabularnewline
\midrule 
Наименование и единица измерения  & Обозначение  & 1  & 2  & 3  & 4  & 5  & 6  & 7  & 8  & 9  & 10\tabularnewline
\midrule 
   \newpage  Положение дозирующей иглы, град  & $(\alpha_{\text{дк}})$  &  &  &  &  &  &  &  &  &  & \tabularnewline
\midrule 
Расход топлива, кг/ч  & $G_{\text{Т}}$  &  &  &  &  &  &  &  &  &  & \tabularnewline
\midrule 
\newpage Исполнитель  & Инженер по испытаниям  & \multicolumn{3}{c|}{%
\begin{minipage}[t]{0.1\columnwidth}%
Начальник БТК ИК%
\end{minipage}} & \multicolumn{4}{>{\centering}p{0.15\paperwidth}|}{%
\begin{minipage}[t]{0.15\columnwidth}%
Начальник участка ИД%
\end{minipage}} & \multicolumn{3}{c|}{%
\begin{minipage}[t]{0.15\columnwidth}%
Представитель заказчика%
\end{minipage}}\tabularnewline
\midrule 
 &  & \multicolumn{3}{c|}{} & \multicolumn{4}{c|}{} & \multicolumn{3}{c|}{}\tabularnewline
\bottomrule 
\end{tabular}
\par\end{flushleft}


\end{document}
